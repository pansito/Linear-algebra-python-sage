\documentclass[12pt]{extarticle}
\usepackage[utf8]{inputenc}
\usepackage[nottoc]{tocbibind}
\usepackage{graphicx}
\usepackage{amssymb}
\usepackage{amsmath}
\usepackage{fancyhdr}
\usepackage{amsfonts}
\usepackage{amssymb}
\usepackage{graphicx}
\usepackage[spanish]{babel}
\usepackage{multicol}
\usepackage{enumitem}
\usepackage[utf8]{inputenc} 
\usepackage{geometry}
\usepackage{float}
\usepackage{cleveref}
 \geometry{
 a4paper,
 total={170mm,257mm},
 left=20mm,
 top=20mm,
 }
\setcounter{MaxMatrixCols}{30}
\renewcommand{\baselinestretch}{1.10}
\newtheorem{theorem}{Theorem}
\newtheorem{acknowledgement}{Acknowledgement}
\newtheorem{algorithm}{Algorithm}
\newtheorem{axiom}[theorem]{Axiom}
\newtheorem{case}{Case}
\newtheorem{claim}{Claim}
\newtheorem{conclusion}{Conclusi?n}
\newtheorem{condition}{Condition}
\newtheorem{conjecture}{Conjecture}
\newtheorem{corollary}[theorem]{Corollary}
\newtheorem{criterion}[theorem]{Criterio}
\newtheorem{definition}[theorem]{Definici\'{o}n}
\newtheorem{example}[theorem]{Ejemplo}
\newtheorem{exercise}[theorem]{Ejercicio}
\newtheorem{lemma}[theorem]{Lemma}
\newtheorem{notation}[section]{Notaci\'{o}n}
\newtheorem{problem}[theorem]{Problema}
\newtheorem{proposition}[theorem]{Proposition}
\newtheorem{Assertion}[theorem]{Assertion}
\newtheorem{remark}[theorem]{Remark}
\newtheorem{solution}[theorem]{Soluci\'{o}n}
\newtheorem{summary}{Summary}
\numberwithin{equation}{section}
\newenvironment{proof}[1][Proof]{\text{#1.} }{\hfill $\square$}
\newcommand{\qsum}{\boxplus}
\newcommand{\R}{\mathbb{R}}
\newcommand{\A}{\mathcal{A}}
\newcommand{\Le}{\mathcal{L}}
\newcommand{\M}{\mathcal{M}}
\newcommand{\Limit}{\lim_{|x|\rightarrow\infty}}
\newcommand{\Dsum}{\displaystyle\sum}
\title{MCA: Multiple Correspondence Analysis}
\author{Pan }
\date{June 2021}

\begin{document}
\maketitle
\section{Abbreviations and Glossary:}
CA: Correspondence Analysis \\
PCA: Principal Component Analysis \\
MCA: Multiple Correspondence analysis \\
SVD: Singular value decomposition \\

\section{Motivation}
Understand and optimize the presentation of the data\\
Being centred on CA and in MCA the introduction to operations\\ 
Find a way to present the data :)fgdfgdfgdfgdfg
ddfgdfgdfgdfgdfgdfgdfgdffgdffgdffgdffgdfgdfg
\section{An introduction to correspondence analysis}
\begin{definition}
The PCA is a method for reducing number of data and identifying dimensions explaining maximum variance in metric data. 
\end{definition}
\begin{definition}
We can understand CA as technique that provides graphical representations of Cross-tables or contingency tables, as an exploratory method based on well-known geometrical paradigms which it's also a type of PCA but applied to categorical data. Both methods are based on the decomposition of a centered and normalized matrix.
Now we have the CA Process:\\
First, we need to use a cross-table or a contingency table, of two variables with $I$ rows and $J$ columns. This table will be denoted by $N$ and it's elements by $n_{ij}$ which means that the element is in the row $i$, and column $j$, with $i = 1,...,I$ and $j = 1,...,J$. \\
The first step is to calculate the called \textit{Correspondence matrix} $P$ which elements are $p_{ij} = \frac{n_{ij}}{n}$ (The relative frequencies), where $n$ is the grand total$\left( \Dsum_{i=1}^I \Dsum_{j=1}^J n_{ij}  \right)$ of  $N$.\\
\end{definition}
Another important elements in CA are the marginal relative frequencies or masses, those are calculated as a row o column sum, as follows: 
$r_{i \cdot} = \displaystyle \sum_{j=1}^J \frac{n_{ij}}{n}$ and $c_{\cdot j} = \displaystyle \sum_{i=1}^I \frac{n_{ij}}{n}$.\\
(those will we be indistinctly with $r_i,c_i$) This elements serve to center and normalize the correspondence matrix.
Since our main goal is to describe or solve an statistical question, it's  important comprehend how CA behaves when we do the hypothesis tests. \\
\textbf{For example:} Under the null hypothesis of Independence (which usually means that the values appear with the same frequency), the expected values of the relative frequencies $p_{ij}$ will be the products of $r_i \cdot c_j$ or the masses. 
Now, for the Centering and normalization process, above mentioned, we use $r_i$ and $c_j$ in the following ways: \\
Centering is the simplest process, just consisting on calculating differences between observed $(p_{ij})$ and expected relative  $r_i \cdot \ c_j$ frequencies. \\
The normalization process consist in dividing the previous differences by the square root of  $r_i \cdot c_j$  which leads to a matrix of standardized residuals: $S_{ij} = \frac{ ( p_{ij} -  r_i \cdot c_j) }{\sqrt{  r_i \cdot c_j}}$

\begin{definition}
A way to describe relationships in contingency tables is the chi-square statistic, which is defined as: 
 $$ \chi^2 = \sum_i \sum_j \dfrac{(n_{ij} - \hat{n}_{ij})^2}{\hat{n}_{ij}} \text{ Where } \hat{n}_{ij} =  \frac{n_{\cdot i} \times n_{\cdot j}}{n}, \{i = 1,...,I; j= 1,...,J\}$$
\\
If we repeat this process for relative frequencies $(p_{ij})$, we obtain the chi-square divided by the grand total: 
$$ \frac{\chi^2}{n} = \sum_i \sum_j \dfrac{(p_{ij} - \hat{p}_{ij})^2}{\hat{p}_{ij}} \text{ Where } \hat{p}_{ij} = r_i \times c_j$$
\\
And this is exactly the total inertia. \\
\end{definition}
\begin{definition}
In the CA it's important to know the differences between the elements in the Cross-table, and this is possible by using the so-called $\chi^2$-distances, which are basically weighted Euclidean distances between normalized rows with weights inversely proportional to the square roots of the column totals. So, between the $i$ and the $k$ row we have:
$$ d_{ik} = \sqrt{\Dsum_{j=1}^J\frac{(\frac{p_{ij}}{r_i} - \frac{p_{kj}}{r_k})^2}{c_j}}$$
This is computes the $(\chi^2 )$ distances using the correspondence matrix %and displays them in a reasonably compact table. 
\end{definition}

It's important to know that this \emph{Chi-squared distances} are equal to the distances between the points represented in principal coordinates in the full space i.e. 
This can be, also, interpreted (in a geometrical way) as the $\emph{weighted sum of squared distances}$ of the column profiles to the average profile. 
\subsection{Reduction of dimensionality}
In the CA/MCA, due to graphical limitations, here we usually use planar maps. Which shows pairs of dimensions at a time, but also are three-dimensional graphical displays can be used. 
It's important to determine the number of dimensions that are going to be interpreted, this can be made in different ways (Similar to PCA): \begin{enumerate}
\item  Consider all those with eigenvalues that explains more  than average inertia.
\item  Examine all ''scree plot'' of eigenvalues to identify the ''elbow'' in the descending sequence.
\item Use the application-based method of including all dimensions that have a coherent substantive interpretation.
\end{enumerate}
\subsection{Computing row scores (SVD)}
Since or goal is to graphically represent  the $\chi^2$-distances we need to find a method to compute that representation. As we mentioned before there's a relation between the $\chi^2$-distance and the position of the dots in the plot representing rows as dots. \\
The usage of singular value decomposition (SVD) provide us a method to compute this distances and create a plot.\\
So, here we can use a algorithm-like process to compute this representation:\\
First we need a matrix of \emph{standardized residuals}, which is exactly the matrix that we create in the normalization process. \\
So we have: \\
$\Omega = \{S_{ij}\} = \frac{ ( p_{ij} -  r_i \cdot c_j) }{\sqrt{  r_i \cdot c_j}}$
This also can be expressed in matrix notation as: $$\Omega =D_r ^{-\frac{1}{2}} (P- r\cdot c ^T)D_c ^{-\frac{1}{2}}$$
where $r, c$ are the vector of row and column masses, $D_r, D_C$ are the diagonal matrices with theses masses on the respective diagonals.
them, we can apply the SVD to our matrix $\Omega$ (or $S$), which is basically find orthogonal matrices $V$ and $W$, together with a diagonal matrix $\Lambda$.  such that $\Omega = V \Lambda W^T, VV^T = WW^T =I$ \\
Where $\Lambda$ is a diagonal matrix with singular values in descending order: $\Lambda =  \begin{pmatrix}
\sigma_1 & \cdots &0 \\
\vdots & \ddots & \vdots \\
 0 & \cdots & \sigma_S\\
\end{pmatrix} $ where $\sigma_1 > \cdots < \sigma_S$ \\
S is the rank of matrix $\Omega$, the columns of $V$, are the so-called \emph{left singular vectors} and those of $W$ \emph{right singular vectors}.
Now, the scores of the rows, are given by: \\
$R=\delta_r V \Lambda$ with \\
$\delta_r = \begin{pmatrix}
\frac{1}{\sqrt{p_{1 \cdot}}} & \cdots &0 \\
\vdots & \ddots & \vdots \\
 0 & \cdots & \frac{1}{\sqrt{p_{I \cdot}}}\\
\end{pmatrix}$
\subsection{Contribution to inertia}
Here, we are going to present an additional way to calculate the total inertia. \\
We can obtain the amount of inertia of a row in a axis, for example, if we want to know the inertia of the row $i$ in the axis $s$, we just have to multiply the mass of the $i$Th row $(r_{i})$ by the squared coordinate of the $i$Th row of the $s$Th dimensions $(f ^2_{is})$, so it would be $r_i \cdot f^2_{is}$, we can also compute the inertia of the row, it will be $r_i \cdot \Dsum_ s f_{is}^2 $, y de aquí The sum of the inertias over all rows and over all dimensions, i.e., $\Dsum_i \Dsum_s r_i \cdot f^2_{is}$. gives the total inertia, the same hold for the columns, so we have the following equality: 
$$ TTi= \Dsum_i \Dsum_s r_i \cdot f^2_{is} = \Dsum_j \Dsum_s c_j \cdot \mathrm{g}^2_{js}$$\\
TTi represents the total inertia.\\
\begin{definition}
We define the principal inertias (eigenvalues) $\lambda_s$ as:  $$\lambda_s = \Dsum_i r_i\cdot f^2_{is} = \Dsum_j c_j\cdot \mathrm{g} ^2_{js}, s = 1,\dots,S$$ 
\end{definition}
With those equations (in  terms of rows and points) we can compute: 
\begin{enumerate}
\item  The contribution from each row or each column to total inertia. This is the amount of variance each row or column contributes to the geometric model as a whole. 
\item Same as item 1, but with respect to single dimensions.
\item  The contribution of each dimension to total inertia, i.e. the explained variance of each dimension. 
\item The contribution of each dimension to the inertia of a row or column, i.e. the explained variance of each dimension to a point. The square roots of these values are often called \emph{factor loading's} because they are also correlation coefficients between the point and the dimension.
\item  The amount of explained variance of the first $S^*$ dimensions to each row or to each column. These coefficients are called \emph{qualities } in CA, known as \emph{Communalities} in PCA
\end{enumerate}
   Now, it's important to carefully monitor  the low relative frequencies in CA.

\subsection{Plotting}
\subsubsection{Rows}
In the previous section, we presented a method for finding distances between rows, even though, it's not very useful in order to present a plot, since it, usually, has a large dimension number.  Instead of that, using the SVD and taking just he first two component of each row's score, we can produce a reasonable approximation to the $\chi^2$-distances.\\
Also, we can use some result from SVD, the principal a standard coordinates, that are very helpful to make this maps. So we have: \\
\begin{align*}
\text{Principal coordinates of rows}:  F = D_r^{-\frac{1}{2}} V\Lambda \\
\text{Standard coordinates of rows}:  A = D_r^{-\frac{1}{2}} V 
\end{align*}
\\
\subsubsection{Columns}
In the same way that we used CA to derive a visual representation of the relationship between rows, we can also use it to illustrate the relationship between columns or between rows and columns.\\
In the same way that the rows, we start deriving scores, but this time for the columns, from the SVD, so we calculate the matrix:\\
$C=\delta_c W$ \\
$\delta_c = \begin{pmatrix}
\frac{1}{\sqrt{p_{ \cdot 1}}} & \cdots &0 \\
\vdots & \ddots & \vdots \\
 0 & \cdots & \frac{1}{\sqrt{p_{ \cdot J}}}\\
\end{pmatrix}$\\ 
With this given, and in the same way that for rows, we can also obtain principal an standard coordinates for columns: 
\begin{align*}
\text{Principal coordinates of columns}:  G = D_c^{-\frac{1}{2}} W\Lambda \\
\text{Standard coordinates of rows}:  B = D_c^{-\frac{1}{2}} W
\end{align*}
\\
In he plotting process of a two-dimensional map we use:
\begin{enumerate}
\item The first two columns of the coordinates matrices $F$ and $G$ for the symmetric map.
\item $A$ and $G$ for the asymmetric map of columns. 
\item $F$ and $B$ for the asymmetric map of the rows
\end{enumerate}
Using any of this, the proportion of inertia explained would be $\frac{(\sigma_1^2 + \sigma_2^2)}{\Dsum_s \sigma_s^2}$ i.e. $\frac{(\lambda_1 + \lambda_2)}{\Dsum_s \lambda_s}$ 
\subsection{Example}
Here we are going to use the cross-table presented on the guide book. 
So first we have: \\

    \begin{table}[h]
    \centering
    \begin{tabular}{c c c c c c c }
    \hline
    \hspace{1pt} & U.K   & U.S. & Russia & Spain & France & Total \\ \hline
       Agree strongly  & 230 & 400& 1010 & 201 & 365 & 2206 \\
       Agree  & 329 & 471 & 530 & 639 & 478 & 2447 \\
       Neither nor & 177 & 237 & 141 & 208 & 305 & 1068 \\
       Disagree & 34 & 28 & 21 & 72 & 50 & 205 \\
       Disagree strongly & 6 & 12 & 11 & 14 & 97 & 140 \\
       Total & 776 & 1148 & 1713 & 1134 & 1295 & 6066  \\
       \hline
    \end{tabular}
    \label{Table 1.1}
\end{table}
\newpage
Also the percentage (per country) table: 
    \begin{table}[H]
    \centering
    \begin{tabular}{c c c c c c c }
    \hline
    \hspace{1pt} & U.K   & U.S. & Russia & Spain & France & Average \\ \hline
       Agree strongly  & 29.6 & 34.8 & 59 &17.7 & 28.2 & 36.4 \\
       Agree  & 42.4 & 41 & 30.9 &56.4 & 36.9 & 40.3 \\
       Neither nor &22.8 &20.6 &8.2 & 18.3 & 23.6 & 17.6 \\
       Disagree &4.4 &2.4 &1.2 &6.5 &3.9 &3.4 \\
       Disagree strongly &0.8 &1.1 &0.6 &1.2 &7.5 &2.3 \\
       Total & 100 & 100 & 100 & 100 & 100 & 100  \\
       \hline
    \end{tabular}
    \caption{}
    \label{Table 1.2}
\end{table}
Now, since in the table 1.1, we can see that the grand total $n=6066$ we can compute the \emph{Correspondence matrix} 
$$P = \begin{pmatrix}
0.0379 & 0.0659 &  0.1665 & 0.0331 & 0.0601 \\
0.0542 & 0.0776 &  0.0873 & 0.1053 & 0.0787 \\
0.0291 & 0.039 & 0.0232 & 0.0342 & 0.0502 \\
0.0056 & 0.0046 & 0.0034 & 0.0118 & 0.0082 \\
0.0009 & 0.0019 & 0.0018 & 0.0023 & 0.0159 \\
\end{pmatrix}$$
\\
If we present it as a table: 
 \begin{table}[h]
    \centering
    \begin{tabular}{c c c c c c c }
    \hline
    \hspace{1pt} & U.K   & U.S. & Russia & Spain & France & Average \\ \hline
       Agree strongly  &0.0379 & 0.0659 &  0.1665 & 0.0331 & 0.0601 & 0.3636 \\
       Agree  & 0.0542 & 0.0776 &  0.0873 & 0.1053 & 0.0787 & 0.4033\\
       Neither nor &0.0291 & 0.039 & 0.0232 & 0.0342 & 0.0502 & 0.176\\
       Disagree &0.0056 & 0.0046 & 0.0034 & 0.0118 & 0.0082 & 0.0337 \\
       Disagree strongly &0.0009 & 0.0019 & 0.0018 & 0.0023 & 0.0159 & 0.023  \\
       Total & 0.12 & 0.1892 & 0.2823 & 0.1869 & 0.2134  &1 \\
       \hline
    \end{tabular}
    \label{Table 1.2}
\end{table}
\end{document}

















